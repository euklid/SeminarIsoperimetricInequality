%%%%%%%%%%%%%%%%%%%%%%%%%%%%%%%%%%%%%%%%%%%%%%%%%%%%%%%%%%%%%%%%%%%%%%%%%%%%%%%%%%%
% THE BEER-WARE LICENSE (Revision 42): %
% <r@twopi.eu> schrieb diese Datei. Solange Sie diesen Vermerk nicht entfernen, %
% können Sie mit dem Material machen, was Sie möchten. Wenn wir uns eines Tages %
% treffen und Sie denken, das Material ist es wert, können Sie mir dafür ein Bier %
% ausgeben. Robert Hemstedt %
%%%%%%%%%%%%%%%%%%%%%%%%%%%%%%%%%%%%%%%%%%%%%%%%%%%%%%%%%%%%%%%%%%%%%%%%%%%%%%%%%%%

\documentclass[12pt,a4paper]{article}
\usepackage[utf8x]{inputenc}
%\usepackage{ucs}
\usepackage[left=2.0cm, right=2.0cm, top=2.0cm, bottom=2.0cm]{geometry}
\usepackage{amsmath}
\usepackage{amsfonts}
\usepackage[ngerman]{babel}
\usepackage{bbm}
\usepackage{amssymb}
\usepackage[amsthm,thmmarks]{ntheorem}
%\usepackage{tabularx}
\usepackage[arrow, matrix, curve]{xy}
\usepackage{graphicx}
% b) Lemma, Satz, Theorem usw.
\makeatletter
\renewtheoremstyle{plain}%
  {\item[\hskip\labelsep \theorem@headerfont ##1\ ##2:]}%
  {\item[\hskip\labelsep \theorem@headerfont ##1~##2~##3:]\mbox{}}
\makeatother

\theoremstyle{plain}
\newtheorem{Theorem}{Theorem}[section]
\newtheorem{Satz}[Theorem]{Satz}
\newtheorem{Prop}[Theorem]{Proposition}
\newtheorem{Lemma}[Theorem]{Lemma}
\newtheorem{Korollar}[Theorem]{Korollar}
\newtheorem{Definition}[Theorem]{Definition}
\newtheorem*{Folgerung}[Theorem]{Folgerung}
\newtheorem*{Behauptung}[Theorem]{Behauptung}
\newtheorem{bez}[Theorem]{Bezeichnung}

\theorembodyfont{\upshape}
\newtheorem{Bemerkung}[Theorem]{Bemerkung}
\newtheorem{Beispiel}[Theorem]{Beispiel}

\newcommand{\herv}[1]{{\emph{\textbf{#1}}}}
\newcommand{\N}{\mathbb{N}}
\newcommand{\R}{\mathbb{R}}
\newcommand{\Z}{\mathbb{Z}}
\newcommand{\Q}{\mathbb{Q}}
\newcommand{\C}{\mathbb{C}}
\newcommand{\cupdot}{\mathbin{\dot{\cup}}}

\def\presuper#1#2%
  {\mathop{}%
   \mathopen{\vphantom{#2}}^{#1}%
   \kern-\scriptspace%
   #2}

\numberwithin{equation}{section}

\newsavebox{\fmbox}
\newenvironment{fmpage}[1]
{\begin{lrbox}{\fmbox}\begin{minipage}{#1}}
{\end{minipage}\end{lrbox}\fbox{\usebox{\fmbox}}}

\linespread{1.05}
\author{Robert Hemstedt \\ \texttt{r@twopi.eu}}
\title{Seminarvortrag Isoperimetrische Ungleichung in der Ebene}
\begin{document}
\maketitle
\section{Motivation}
Sei $\Gamma$ eine \textbf{geschlossene Kurve} in der Ebene, ohne Selbstüberschneidung. Es bezeichne $l$ die \textbf{Länge} von $\Gamma$ und $\mathcal{A}$ die \textbf{Fläche} der beschränkten Umgebung in $\R^2$, die von $\Gamma$ umschlossen wird.\\
\textit{Frage:} Falls existent, welche Kurve $\Gamma$ für ein festes $l$ maximiert $\mathcal{A}$?

Man kann sich schnell selbst davon überzeugen, dass der Kreis dieses Problem löst.
Wir wollen dies formal beweisen.

\section{Kurven, Längen und Flächen}
Bei der ersten Beschreibung des Problems haben wir die uns alltäglichen Begriffe \textbf{geschlossene Kurve}, \textbf{Länge} und \textbf{Fläche} verwendet, ohne sie vorher klar definiert zu haben. Das holen wir jetzt nach.
\begin{Definition}
Eine \herv{parametrisierte Kurve} ist eine Abbildung \[ \gamma:[a,b] \rightarrow \R^2 .\]
$\operatorname{Im}(\gamma)$ ist eine Menge von Punkten in der Ebene, die wir als \herv{Kurve} $\Gamma$ bezeichnen.\\
Eine Kurve heißt \herv{einfach}, wenn sie sich nicht selbst schneidet und sie heißt \herv{geschlossen}, wenn ihr Anfangs- und Endpunkt identisch sind. Also:
\[ \Gamma \text{ einfach und geschlossen} :\Leftrightarrow \left\lbrace \begin{array}{ll}
\gamma(s_1)=\gamma(s_2)& s_1=a, s_2=b  \\ \forall s_1\neq s_2 \in [a,b]:  \gamma(s_1)\neq \gamma(s_2) & \text{sonst} \end{array}\right. \]
\end{Definition}
\begin{Bemerkung}
Wir können $\gamma$ als eine periodische Funktion auf $\R$ mit Periodenlänge $b-a$ fortsetzen und sie als Funktion auf dem Kreis betrachten.\\
Für unsere weiteren Betrachtungen fordern wir eine gewisse Glattheit von $\gamma$ voraus, sodass wir sie als $\mathcal{C}^1$ Funktion betrachten mit $\gamma'(s)\neq 0\ \forall s$, also $\gamma$ nie konstant ist. \\
Insgesamt garantieren uns diese Forderungen an $\gamma$, dass $\Gamma$ an jedem Punkt eine wohldefinierte Tangente hat, die sich stetig ändert, wenn der Stützpunkt auf der Kurve (stetig) wandert.
\end{Bemerkung}
\begin{Bemerkung}
Die Parametrisierung von $\gamma$ induziert eine \textbf{Orientierung} auf $\Gamma$, wenn $s$ von $a$ nach $b$ geht. Weiterhin ergibt sich für jede bijektive Abbildung $s: [c,d] \rightarrow [a,b]$, $s\in \mathcal{C}^1$ eine neue Parametrisierung $\eta:[c,d] \rightarrow \R^2$ von $\Gamma$ mit \[ \eta(t) = \gamma(s(t)).\]
Es sollte klar sein, dass $\Gamma$ geschlossen und einfach unabhängig von der gewählten Parametrisierung ist.
\end{Bemerkung}
\begin{Definition}
Mit den Bezeichnungen von oben sind die zwei Parametrisierungen $\eta$ und $\gamma$ \herv{äquivalent}, wenn $s'(t)>0$ für alle t, d.h. $\eta$ und $\gamma$ induzieren die gleiche Orientierung auf $\Gamma$.\\
Gilt jedoch $s'(t)<0$ für alle t, so kehrt $\eta$ die Orientierung um.
\end{Definition}
\begin{Definition}
Wird die Kurve $\Gamma$ durch eine Funktion $\gamma(s)=(x(s),y(s))$ parametrisiert, dann ist ihre \herv{Länge} l definiert durch \[
l:=\int_a^b{|\gamma'(s)|ds}=\int_a^b {\left((x'(s)^2+y'(s)^2)\right)^{1/2}ds}.\]
\end{Definition}
\begin{Satz}
Die Länge einer Kurve $\Gamma$ ist unabhängig von deren Parametrisierung.
\end{Satz}
\begin{proof}
Seien $\gamma$ und $\eta$ zwei Parametrisierung mit $\gamma(s(t))=\eta(t)$ wie oben. Dann \[ \int_a^b{\gamma'(s)|ds} = \int_c^d{|\gamma'(s(t))||s'(t)|dt} = \int_c^d{|\eta'(t)|dt} ,\] wobei wir die Kettenregel auf $\gamma$ angewandt und die Variable im Integral substituiert haben.
\end{proof}
Für den anvisierten Beweis wählen wir eine besondere Parametrisierung von $\Gamma$.
\begin{Definition}
Wir bezeichen $\gamma$ als eine \herv{Parametrisierung nach der Bogenlänge}, wenn $|\gamma'(s)=1|$ für alle s.
\end{Definition}

\end{document}
