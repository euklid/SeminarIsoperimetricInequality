%%%%%%%%%%%%%%%%%%%%%%%%%%%%%%%%%%%%%%%%%%%%%%%%%%%%%%%%%%%%%%%%%%%%%%%%%%%%%%%%%%%
% THE BEER-WARE LICENSE (Revision 42): %
% <r@twopi.eu> schrieb diese Datei. Solange Sie diesen Vermerk nicht entfernen, %
% können Sie mit dem Material machen, was Sie möchten. Wenn wir uns eines Tages %
% treffen und Sie denken, das Material ist es wert, können Sie mir dafür ein Bier %
% ausgeben. Robert Hemstedt %
%%%%%%%%%%%%%%%%%%%%%%%%%%%%%%%%%%%%%%%%%%%%%%%%%%%%%%%%%%%%%%%%%%%%%%%%%%%%%%%%%%%

\documentclass[12pt,a4paper]{article}
\usepackage[utf8x]{inputenc}
%\usepackage{ucs}
\usepackage[left=2.0cm, right=2.0cm, top=2.0cm, bottom=2.0cm]{geometry}
\usepackage{amsmath}
\usepackage{amsfonts}
\usepackage[ngerman]{babel}
\usepackage{bbm}
\usepackage{amssymb}
\usepackage[amsthm,thmmarks]{ntheorem}
%\usepackage{tabularx}
\usepackage[arrow, matrix, curve]{xy}
\usepackage{graphicx}
% b) Lemma, Satz, Theorem usw.
\makeatletter
\renewtheoremstyle{plain}%
  {\item[\hskip\labelsep \theorem@headerfont ##1\ ##2:]}%
  {\item[\hskip\labelsep \theorem@headerfont ##1~##2~##3:]\mbox{}}
\makeatother

\theoremstyle{plain}
\newtheorem{Theorem}{Theorem}[section]
\newtheorem{Satz}[Theorem]{Satz}
\newtheorem{Prop}[Theorem]{Proposition}
\newtheorem{Lemma}[Theorem]{Lemma}
\newtheorem{Korollar}[Theorem]{Korollar}
\newtheorem{Definition}[Theorem]{Definition}
\newtheorem*{Folgerung}[Theorem]{Folgerung}
\newtheorem*{Behauptung}[Theorem]{Behauptung}
\newtheorem{bez}[Theorem]{Bezeichnung}

\theorembodyfont{\upshape}
\newtheorem{Bemerkung}[Theorem]{Bemerkung}
\newtheorem{Beispiel}[Theorem]{Beispiel}

\newcommand{\herv}[1]{{\emph{\textbf{#1}}}}
\newcommand{\N}{\mathbb{N}}
\newcommand{\R}{\mathbb{R}}
\newcommand{\Z}{\mathbb{Z}}
\newcommand{\Q}{\mathbb{Q}}
\newcommand{\C}{\mathbb{C}}
\newcommand{\cupdot}{\mathbin{\dot{\cup}}}

\def\presuper#1#2%
  {\mathop{}%
   \mathopen{\vphantom{#2}}^{#1}%
   \kern-\scriptspace%
   #2}

\numberwithin{equation}{section}

\newsavebox{\fmbox}
\newenvironment{fmpage}[1]
{\begin{lrbox}{\fmbox}\begin{minipage}{#1}}
{\end{minipage}\end{lrbox}\fbox{\usebox{\fmbox}}}

\linespread{1.05}
\author{Robert Hemstedt \\ \texttt{r@twopi.eu}}
\title{Seminarvortrag Isoperimetrische Ungleichung in der Ebene}
\begin{document}
\maketitle
\section{Motivation}
Sei $\Gamma$ eine \textbf{geschlossene Kurve} in der Ebene, ohne Selbstüberschneidung. Es bezeichne $l$ die \textbf{Länge} von $\Gamma$ und $\mathcal{A}$ die \textbf{Fläche} der beschränkten Umgebung in $\R^2$, die von $\Gamma$ umschlossen wird.\\
\textit{Frage:} Falls existent, welche Kurve $\Gamma$ für ein festes $l$ maximiert $\mathcal{A}$?

Man kann sich schnell selbst davon überzeugen, dass der Kreis dieses Problem löst.
Wir wollen dies formal beweisen.

\section{Kurven, Längen und Flächen}
Bei der ersten Beschreibung des Problems haben wir die uns alltäglichen Begriffe \textbf{geschlossene Kurve}, \textbf{Länge} und \textbf{Fläche} verwendet, ohne sie vorher klar definiert zu haben. Das holen wir jetzt nach.
\begin{Definition}
Eine \herv{parametrisierte Kurve} ist eine Abbildung \[ \gamma:[a,b] \rightarrow \R^2 .\]
$\operatorname{Im}(\gamma)$ ist eine Menge von Punkten in der Ebene, die wir als \herv{Kurve} $\Gamma$ bezeichnen.\\
Eine Kurve heißt \herv{einfach}, wenn sie sich nicht selbst schneidet und sie heißt \herv{geschlossen}, wenn ihr Anfangs- und Endpunkt identisch sind. Also:
\[ \Gamma \text{ einfach und geschlossen} :\Leftrightarrow \left\lbrace \begin{array}{ll}
\gamma(s_1)=\gamma(s_2)& s_1=a, s_2=b  \\ \forall s_1\neq s_2 \in [a,b]:  \gamma(s_1)\neq \gamma(s_2) & \text{sonst} \end{array}\right. \]
\end{Definition}
\begin{Bemerkung}
Wir können $\gamma$ als eine periodische Funktion auf $\R$ mit Periodenlänge $b-a$ fortsetzen und sie als Funktion auf dem Kreis betrachten.\\
Für unsere weiteren Betrachtungen fordern wir eine gewisse Glattheit von $\gamma$ voraus, sodass wir sie als $\mathcal{C}^1$ Funktion betrachten mit $\gamma'(s)\neq 0\ \forall s$, also $\gamma$ nie konstant ist. \\
Insgesamt garantieren uns diese Forderungen an $\gamma$, dass $\Gamma$ an jedem Punkt eine wohldefinierte Tangente hat, die sich stetig ändert, wenn der Stützpunkt auf der Kurve (stetig) wandert.
\end{Bemerkung}
\begin{Bemerkung}
Die Parametrisierung von $\gamma$ induziert eine \textbf{Orientierung} auf $\Gamma$, wenn $s$ von $a$ nach $b$ geht. Weiterhin ergibt sich für jede bijektive Abbildung $s: [c,d] \rightarrow [a,b]$, $s\in \mathcal{C}^1$ eine neue Parametrisierung $\eta:[c,d] \rightarrow \R^2$ von $\Gamma$ mit \[ \eta(t) = \gamma(s(t)).\]
Es sollte klar sein, dass $\Gamma$ unabhängig von der gewählten Parametrisierung  geschlossen und einfach ist.
\end{Bemerkung}
\begin{Definition}
Mit den Bezeichnungen von oben sind die zwei Parametrisierungen $\eta$ und $\gamma$ \herv{äquivalent}, wenn $s'(t)>0$ für alle t, d.h. $\eta$ und $\gamma$ induzieren die gleiche Orientierung auf $\Gamma$.\\
Gilt jedoch $s'(t)<0$ für alle t, so kehrt $\eta$ die Orientierung um.
\end{Definition}
\begin{Definition}
Wird die Kurve $\Gamma$ durch eine Funktion $\gamma(s)=(x(s),y(s))$ parametrisiert, dann ist ihre \herv{Länge} l definiert durch \[
l:=\int_a^b{|\gamma'(s)|ds}=\int_a^b {\left((x'(s)^2+y'(s)^2)\right)^{1/2}ds}.\]
\end{Definition}
\begin{Satz}
Die Länge einer Kurve $\Gamma$ ist unabhängig von deren Parametrisierung.
\end{Satz}
\begin{proof}
Seien $\gamma$ und $\eta$ zwei Parametrisierung mit $\gamma(s(t))=\eta(t)$ wie oben. Dann \[ \int_a^b{|\gamma'(s)|ds} = \int_c^d{|\gamma'(s(t))||s'(t)|dt} = \int_c^d{|\eta'(t)|dt} ,\] wobei wir die Kettenregel auf $\gamma$ angewandt und die Variable im Integral substituiert haben.
\end{proof}
Für den anvisierten Beweis wählen wir eine besondere Parametrisierung von $\Gamma$.
\begin{Definition}
Wir bezeichen $\gamma$ als eine \herv{Parametrisierung nach der Bogenlänge}, wenn $|\gamma'(s)|=1$ für alle s.
\end{Definition}
Das bedeutet, dass $\gamma(s)$ sich mit konstanter Geschwindigkeit bewegt und daher die Länge von $\Gamma$ genau $b-a$ ist. Nach einer eventuellen Translation lässt sich die Parametrisierung auf $[0;l]$ definieren.
\begin{Satz}
Jede Kurve mit den bisherigen Voraussetzungen ($\gamma\in\mathcal{C}^1, \gamma'(s)\neq 0 \forall s$) lässt sich nach der Bogenlänge parametrisieren.
\end{Satz}
\begin{proof}
O.b.d.A sei $\alpha(t):[0,b]\rightarrow \R^2$ eine Parametrisierung der Kurve $\Gamma$. Dann definiere $L(s):=\int_0^s{|\alpha'(t)|dt}$. Da $|\alpha'(t)| > 0$ für alle $t$ und $\alpha'$ stetig, ist $L$ streng monoton und damit eine Bijektion $L:[0,b]\rightarrow [0,l]$. Dann ist $\gamma(s)=\alpha(L^{-1}(s))$ eine Parametrisierung von $\Gamma$ nach der Bogenlänge. In der Tat: \[\gamma'(s)=\frac{d}{ds}\alpha(L^{-1}(s))=\alpha'({L^{-1}(s)})\cdot (L^{-1})'(s) = \alpha'(L^{-1}(s))\cdot \frac{1}{L'(L^{-1}(s))}= \frac{\alpha'(L^{-1}(s))}{|\alpha'(L^{-1}(s))|}\] und somit ist $|\gamma'(s)|=1$ für alle s.
\end{proof}
Wir wollen nun die Fläche der von der einfach geschlossenen Kurve $\Gamma$ umschlossenen definieren. Für unsere Betrachtungen genügt sie der folgenden Formel:
\begin{Definition}
Der Flächeninhalt $\mathcal{A}$ der von der Kurve $\Gamma$ umschlossenen Region lässt sich wie folgt berechnen:
\begin{align*}
\mathcal{A}&=\frac{1}{2}\left|\int_\Gamma (x dy -y dx) \right| \\
&= \frac{1}{2}\left|\int_a^b x(s)y'(s)-y(s)x'(s) ds \right| .
\end{align*}
\end{Definition}
\begin{Bemerkung}
Für einen einfacheren Fall lässt sich folgende Gleichung herleiten: \[
\mathcal{A}=\left|-\int_\Gamma{y dx}\right|=\int_0^1{f(x)dx} - \int_0^1{g(x)dx}, \] wobei eine Gerade existiert, die $\Gamma$ in zwei $\mathcal{C}^1$-Funktionen $f,g: [0,1]\rightarrow \R$ über dieser Geraden als $x$-Achse zerfallen lässt, wobei $f(x)\geq g(x)$. Wenn der umschlossene Bereich $\Omega=\{(x,y):0\leq x\leq 1, g(x)\leq y\leq f(x)\}$ stets \glqq links\grqq von $\Gamma$ liegt, können die Betragsstriche entfallen.
\end{Bemerkung}
\begin{proof}
O.B.d.A sei $\gamma:[0,2] \rightarrow \R^2, s\mapsto \left\lbrace \begin{array}{ll}\left( \begin{array}{c} s \\ g(s) \end{array}\right)& 0\leq s\leq 1\\
\left(\begin{array}{c} 2-s \\ f(2-s)\end{array}\right) & 1\leq s \leq 2 \end{array}\right.$.\\ Es ist leicht nachgerechnet, dass dies eine wohldefinierte Parametrisierung von $\Gamma$ ist, wobei $\gamma\in\mathcal{C}^1$ und $\gamma'(s)\neq 0 \forall s$ und $\Omega$ liegt immer \glqq linksseitig \grqq. Es folgt:
\begin{align*}
-\int_\Gamma {ydx}&=-\left(\int_0^1{g(x(s))\cdot x'(s)ds} + \int_1^2{f(x(s))\cdot x'(s)ds } \right) \text{ mit } x(s)=\left\lbrace\begin{array}{ll} s & 0\leq s\leq 1 \\ 2-s & 1\leq s \leq 2 \end{array}\right.\\
&= -\left(\int_0^1{g(x)dx} + \int_1^0{f(x)dx}\right) = -\left(\int_0^1{g(x)dx} - \int_0^1{f(x)dx}\right)\\
&=\int_0^1{f(x)dx} - \int_0^1{g(x)dx}
\end{align*}
Wird $\gamma(s)$ mit entgegengesetzter Orientierung definiert, also erst $\operatorname{Im}(f(x))$ und dann $\operatorname{Im}(g(x))$ durchlaufen, müssen Betragsstriche gesetzt werden.
\end{proof}
\section{Formulierung und Beweis der Isoperimetrischen Ungleichung}
An dieser Stelle sind wir mit genügend Rüstzeug bewappnet, um unser Problem exakt zu formulieren und zu beweisen.
\begin{Theorem}[Isoperimetrische Ungleichung]
Sei $\Gamma$ eine einfache geschlossene Kurve im $\R^2$ der Länge l und sei $\mathcal{A}$ der Flächeninhalt der ihr umschlossenen Region. Dann gilt
\[ A\leq \frac{l^2}{4\pi}, \]
Gleichheit gilt genau dann, wenn $\Gamma$ ein Kreis ist.
\end{Theorem}
\begin{proof}
Wir können unser Problem neu skalieren, indem wir die Maßeinheiten um einen Faktor $\delta>0$ verändern. Dies definiert die Abbildung $(x,y) \mapsto (\delta x, \delta y)$. Unter dieser Abbildung hat die Kurve $\Gamma$ die Länge $\delta l$ und die Fläche $\delta^2 \mathcal{A}$. Wir können somit alle Fälle auf den Fall $l=2\pi$ zurückführen und es genügt zu zeigen, dass $\mathcal{A}\leq \pi$ mit Gleichheit genau dann, wenn $\Gamma$ ein Kreis ist.

Sei $\gamma:[0,2\pi]\rightarrow \R^2$ mit $\gamma(s)=(x(s),y(s))$ eine Parametrisierung nach der Bogenlänge der Kurve $\Gamma$, also $x'(s)^2+y'(s)^2=1$ für alle $s\in [0,2\pi]$. Dann \begin{align}\label{paramlaenge}
\frac{1}{2\pi} \int_0^{2\pi} {\left(x'(s)^2+y'(s)^2\right) ds} = 1.
\end{align} 
Da die Kurve geschlossen und die Funktionen $x(s)$ und $y(s)$ $2\pi$-periodisch ist, betrachten wir ihre Fourierreihen \[x(s)\sim \sum a_ne^{ins} \text{ und } y(s)\sim \sum b_ne^{ins}. \]  Weiter mit $\hat{f'}(n)=i n\hat{f}(n)$ haben wir \[x'(s)\sim \sum  a_n i n e^{ins} \text{ und } y'(s)\sim \sum b_n i n e^{ins}. \] Die Parsevalsche Identität ($\frac{1}{2\pi}\int_0^{2\pi}|f(\theta)|^2d\theta = \sum_{n=-\infty}^\infty |a_n|^2$) angewandt auf \eqref{paramlaenge} gibt \begin{align} \label{parseval1}
\sum_{n=-\infty}^\infty |n|^2(|a_n|^2+|b_n|^2)=1.
\end{align}
Mit der Bilinearform der Parsevalschen Identität angewandt auf das Integral für den Flächen\-in\-halt erhalten wir mit $a_n=\overline{a_{-n}}$ und $b_n=\overline{b_{-n}}$, da $x(s)$ und $y(s)$ reellwertig sind:
\[\mathcal{A}=\frac{1}{2} \left|\int_0^{2\pi} {x(s)y'(s)-y(s)x'(s)ds}\right|=\pi\left|\sum_{n=-\infty}^\infty {n\left(a_n\overline{b_n}-b_n\overline{a_n}\right) } \right|. \] 
Ferner ist \begin{align} \label{abschaetz}
|a_n\overline{b_n} -b_n\overline{a_n}|\leq |a_n\overline{b_n}|+|\overline{a_n}b_n|=2|a_n||b_n|\leq |a_n|^2+|b_n|^2
\end{align}
Mit $|n|\leq |n|^2$ und \eqref{parseval1} zusammen ergeben \begin{align*}
\mathcal{A}&\leq \pi \sum_{n=-\infty}^{\infty}{|n|^2(|a_n|^2+|b_n|^2)} \\
&\leq \pi
\end{align*}
Wenn $\mathcal{A}=\pi$, dann muss $x(s)=a_{-1}e^{-is}+a_0+a_1e^{is}$ und $y(s)=b_{-1}e^{-is}+b_0+b_1e^{is}$, da $|n|<|n|^2,$ sobald $|n|\geq 2$. Da $x(s)$ und $y(s)$ reellwertig sind, muss $a_{-1}=\overline{a_1}$ und $b_{-1}=\overline{b_1}$. Aus \eqref{parseval1} ist $2(|a_1|^2+|b_1|^2)=1$ und wegen Gleichheit in \eqref{abschaetz} ist $|a_1|=|b_1|=1/2.$ Demnach $a_1=\frac{1}{2}e^{i\alpha}$ und $b_1=\frac{1}{2}e^{i\beta}$. Weiter ist $1=2|a_1\overline{b_1}-\overline{a_1}b_1|=2\left|\frac{e^{i(\alpha-\beta)}}{4}-\frac{e^{i(\beta-\alpha)}}{4}\right|=|\sin(\alpha - \beta)|$ und somit $\alpha-\beta=(2k+1)\pi/2, k\in \Z$. Damit finden wir
\[ x(s)=a_0+\cos(\alpha+s) \text{ und } y(s)=b_0 \pm\sin(\alpha +s),\] wobei das Vorzeichen in $y(s)$ von der Parität von $k$ abhängt. Auf jeden Fall sehen, wir dass $\Gamma$ ein Kreis ist, für den die Gleichheit offensichtlich gilt und damit ist die Aussage bewiesen.
\end{proof}
\section{Diskussion}
Wir bemerken:
\begin{enumerate}
\item Wir haben vorausgesetzt, dass $\Gamma$ eine einfache, geschlossene Kurve ist und an jeder Stelle differenzierbar. Tatsächlich funktioniert der Beweis, so wie er formuliert ist auch wenn $\Gamma$ nicht einfach, aber geschlossen und an jeder Stelle differenzierbar ist.
\item Es gibt einen Beweis von T. Bonnesen, der die isoperimetrische Ungleichung auf elementargeometrische Weise löst. Dabei wird vorausgesetzt, dass $\Gamma$ ein konvexes Polygon ist. Man sieht leicht ein, dass die Gleichung, wenn sie für das konvexe Polygon gelten sollte auch auf jeden Fall für die konkaven Varianten gilt, da diese einen kleineren Flächeninhalt haben, da beim konkaven Polygon mit den gleichen Seitenlängen, einige Seiten \glqq nach innen geklappt\grqq\ sind.
\item Wir haben für die Flächeninhaltsberechnung unsere Formel vorausgesetzt, ohne zu klären, ob es noch andere Möglichkeiten gibt, den Flächeninhalt zu definieren.
\end{enumerate}
\section{Übungsaufgabe}

\end{document}
